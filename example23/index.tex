\documentclass{article}
\usepackage{xcolor}
\pagecolor[rgb]{0,0,0}
\color[rgb]{1,1,1}
\begin{document}
\section*{Quadratic equations}
\begin{equation}
  \label{quad}
  ax^2 + bx + c = 0
\end{equation}
\begin{equation}
  \label{root}
  x_{1,2} = \frac{-b \pm \sqrt{b^2-4ac}}{2a}.
\end{equation}
If the \emph{discrimimant} \(\Delta \) with
\[
  \Delta = b^2-4ac
\]
Is zero, then the equation (\ref{quad}) has a double solution:
(\ref{root}) becomes
\[
  x = - \frac{b}{2a}
\]

\[ x_1^2 + x_2^2 = 1, \quad 2^{2^x} = 64  \]

\[ \sqrt[64]{x} = \sqrt{\sqrt{\sqrt{\sqrt{\sqrt{\sqrt{x}}}}}} \]

\[ \frac{n(n+1)}{2}, \quad \frac{\frac{\sqrt{x}+1}{2}-x}{y^2} \]

\[ \mathcal{A}, \mathcal{B}, \mathcal{C}, \ldots, \mathcal{Z} \]

\end{document}